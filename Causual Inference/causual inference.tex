% Tipo di documento
\documentclass[10pt, a4paper, twoside]{article}
    \usepackage[italian]{babel}
    \usepackage[utf8]{inputenc}
    \usepackage[T1]{fontenc}
%%%%% Pacchetti
\usepackage{FileAusiliari/Layout}			% Contiene i pacchetti e le impostazioni per il layout
\usepackage{FileAusiliari/Pacchetti}		% Pacchetti aggiuntivi di vario tipo (senza tikz)
\usepackage{FileAusiliari/TikZ}				% Ambiente tikzpicture
\usepackage{FileAusiliari/Definizioni}		% Definizioni di colori, variabili globali ecc.
\usepackage{FileAusiliari/Environments}		% Impostazioni TOC, bibliografia e indice analitico + environments vari per il contenuto del documento
\usepackage{FileAusiliari/Custom}			% Tutto ciò che è personalizzabile normalmente dall'utente (tranne i colori per collegamenti ipertestuali, citazioni, link, che sono da modificare in Referencing)
\usepackage{FileAusiliari/Referencing}		% Collegamenti ipertestuali e indice analitico
\usepackage{FileAusiliari/Comandi}			% Comandi vari
%%%%%%%%%%%%%%%%%%%%%%%%%%%%
%%%%%%%%%%%%%%%%%%%%%%%%%%%%

\begin{document}
\title{Casual Inference}
\author{Kirby CHEN\\chen.12712@buckeyemail.osu.edu.}
% \date{\today}
\maketitle
% \begin{BoxTitolo}[colbacktitle=black]{
% Abstract}{Introduzione}\end{BoxTitolo}

\vspace{.5cm}
\pagestyle{fancy}
\pagestyle{fancyfront}

\tableofcontents
\pagestyle{fancymain}
%%%%%%%%%%%%%%%%%%%%%%%%%%%%
%%%%%%%%%%%%%%%%%%%%%%%%%%%%

\input{Sezioni/Sezione1.tex}
\input{Sezioni/Sezione2.tex}
\input{Sezioni/Sezione3.tex}

%%%%%%%%%%%%%%%%%%%%%%%%%%%%
%%%%%%%%%%%%%%%%%%%%%%%%%%%%
% TOC Appendici
\afterpage{\titlecontents{section}[-1pc]
{\addvspace{0pt}}
{\begin{tikzpicture}
		\pgftext{\large\bfseries\sc\bfseries\color{black} \thecontentslabel{\color{white}..}};
	\end{tikzpicture}}
{}
{\color{black}\dotfill\;\;\large\sc\bfseries\thecontentspage}}

\pagestyle{fancy}
% \begin{appendix}
% 	\section{Appendice $1$}	\blindduck[maths]
% \end{appendix}


\pagestyle{empty}


\afterpage{\fancyhead[RE,LO]{\bf Riferimenti bibliografici}}
\vspace{1cm}
\titleformat{\section}
	[hang]{\normalfont}{}{0em}{}
	\normalfont{\Large\sc\bfseries\underline{Riferimenti bibliografici}}
	\vspace{-1cm}
	\normalfont
	\nocite{*}
\bibliographystyle{amsalpha}
\renewcommand\refname{}

\begin{thebibliography}{10}
\input{FileAusiliari/Bibliografia.bib}
\end{thebibliography}
\addcontentsline{toc}{section}{Bibliografia}
\end{document}
